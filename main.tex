\documentclass[twocolumn]{article}
\usepackage{cite}
\PassOptionsToPackage{hyphens}{url}
\usepackage{hyperref}
\usepackage{graphicx}
\usepackage{tabularx}
\usepackage{booktabs}
\usepackage{authblk}
\usepackage{abstract}

\title{\textbf{Mapping the Intellectual Structure of Social Network Research: A Bibliometric Analysis of Three Journals}}
\author{Pengjia Cui \thanks{Corresponding author: pcui@ucsd.edu}}
\affil{Computational Social Science \\
	University of California San Diego}
\date{}

\makeatletter
\renewenvironment{abstract}{%
	\small%
	\begin{center}%
		\bfseries \abstractname\vspace{-.5em}\vspace{\z@}%
	\end{center}%
	\quotation
}{\endquotation}
\makeatother

%%%%%%%%%%%%%%%%%%%%%%%%%%%%%%%%%%%%%%%%%%%%%%%%%%%%%%%%%%%%%%%%%%%%%%%%%%%%%%%%%%%%%%%%%%%%%%%%%%%%%%%%%%

\begin{document}
	
	\twocolumn[
	\maketitle
	\begin{onecolabstract}
		Understanding the intellectual landscape of social network research is crucial for tracking its theoretical and methodological evolution. This study conducts a comprehensive bibliometric analysis of research published in three leading journals in the field: \textit{Social Networks}, \textit{Journal of Complex Networks}, and \textit{Network Science}. These journals were selected due to their distinct aims and scopes: \textit{Social Networks} focuses on sociological and empirical studies, \textit{Journal of Complex Networks} emphasizes mathematical and computational approaches, and \textit{Network Science} bridges interdisciplinary perspectives in network theory. This diversity reflects the inherently interdisciplinary nature of network science. Using data retrieved from the \textit{Web of Science} (WoS), we apply \textbf{performance analysis} to evaluate publication and citation trends, \textbf{science mapping} to identify thematic clusters and knowledge structures, and \textbf{network analysis} to examine collaboration patterns among authors and institutions. Our findings reveal key thematic shifts, influential publications, and the evolving structure of scholarly collaboration. Additionally, we highlight the increasing differentiation between empirical and computational approaches to social network research. By integrating multiple bibliometric techniques, this study provides a structured and data-driven overview of the field’s development, offering insights into emerging research areas and future directions.
	\end{onecolabstract}
	\vspace{1cm}
	]

	\section{Introduction}\label{Introduction}
	
	Bibliometric analysis provides a systematic method for uncovering the intellectual landscape and research dynamics within a scientific field. It enables the identification of key research themes, methodological trends, and the evolution of scholarly discourse. As an inherently interdisciplinary domain, network science—spanning sociology, computer science, mathematics, and physics—poses unique challenges for bibliometric studies due to its methodological and application diversity.
	
	This study analyzes three leading journals in the field: \textit{Social Networks}, \textit{Journal of Complex Networks}, and \textit{Network Science}. These journals were selected for their distinct scopes: \textit{Social Networks} emphasizes sociological and empirical studies, \textit{Journal of Complex Networks} focuses on mathematical and computational approaches, and \textit{Network Science} bridges interdisciplinary perspectives. By examining these journals, we aim to provide a comprehensive overview of the intellectual structure, thematic evolution, and collaboration patterns within social and complex network research.
	
	Our objectives are to: (1) map the intellectual structure of the field via citation networks and scholarly influences, (2) identify key themes and trends through keyword co-occurrence and thematic clustering, (3) evaluate research impact by analyzing citations, influential publications, and prolific contributors, and (4) explore collaboration patterns through co-authorship and institutional networks.
	
	To achieve these goals, we employ a three-pronged approach. First, we conduct a performance analysis to assess publication trends, citation impacts, and contributions of leading scholars and institutions. Second, we utilize science mapping techniques, including co-citation analysis, bibliographic coupling, and keyword co-occurrence, to uncover thematic clusters and intellectual structures. Third, we perform network analysis to examine collaboration patterns across authors, institutions, and countries.
	
	By integrating these methods, this study offers a structured, data-driven overview of the research landscape in social and complex network studies. It highlights the field's development, emerging trajectories, and potential future directions.
	
	
	\section{Bibliometric Techniques}\label{Bibliometric Techniques}
	
	In Table \ref{table.tab2}, we provide a comprehensive comparison of key metrics across the three journals, including their publication frequency, impact factors, citation dynamics, and collaborative patterns. This table highlights both the similarities and distinct characteristics of each journal, such as their publisher affiliations, the types of contributions they attract, and the thematic focus of their most cited works. By examining these metrics, we aim to offer a clear overview of the journals' relative influence and roles within the field of social and complex network research.
	
	\begin{table*}[htbp]
		\tiny
		\caption{Comparison of Metrics}\label{table.tab2}
		\begin{tabularx}{\textwidth}{|p{5cm}|X|X|X|}
			\hline
			\textbf{Category} & \textbf{Social Networks} & \textbf{Network Science} & \textbf{Journal of Complex Networks} \\ \hline
			\textbf{Publisher} & Elsevier & Cambridge University Press & Oxford University Press \\ \hline
			\textbf{Publication Frequency} & 4 issues/year & 4 issues/year & 6 issues/year \\ \hline
			\textbf{Impact Factor (2023)} & 2.9 & 1.4 & 2.2 \\ \hline
			\textbf{5-Year Impact Factor} & 3.1 & 1.7 & 2.0 \\ \hline
			\textbf{Eigenfactor Score} & 0.00387 & 0.00084 & 0.00160 \\ \hline
			\textbf{Article Influence Score} & 1.368 & 0.741 & 0.692 \\ \hline
			\textbf{Cited Half-life (years)} & 16.3 & 7.4 & 5.7 \\ \hline
			\textbf{JCI (2023)} & 1.89 & 0.66 & 0.63 \\ \hline
			\textbf{Immediacy Index} & 0.8 & 0.2 & 0.5 \\ \hline
			\textbf{Total Citable Items (2023)} & 240 & 99 & 163 \\ \hline
			\textbf{Open Access Percentage} & 31.67\% & 44.44\% & 22.09\% \\ \hline
			\textbf{Top Contributing Organization} & University of Oxford (14) & University of California (12) & Santa Fe Institute (7) \\ \hline
			\textbf{Top Contributing Country} & USA (103 papers) & USA (59 papers) & USA (39 papers) \\ \hline
			\textbf{Total Citations (2023)} & 6,795 & 856 & 1,117 \\ \hline
			\textbf{Self-citations (\%)} & 5.55\% & 3.62\% & 3.58\% \\ \hline
			\textbf{Top Citation Source} & Social Networks (43) & Social Networks (34) & Physica A (35) \\ \hline
			\textbf{Top Cited Source} & Social Networks (377) & PNAS (51) & Physical Review E (159) \\ \hline
			\textbf{International Collaboration (\%)} & 47.2\% & 51.3\% & 45.6\% \\ \hline
			\textbf{Top Collaborating Countries} & USA-England & USA-Germany & USA-China \\ \hline
			\textbf{High-frequency Keywords} & Social capital, community & Network topology, algorithms & COVID-19, resilience \\ \hline
			\textbf{Emerging Topics} & Computational sociology & Complex systems & Epidemic modeling \\ \hline
			\textbf{Annual Growth Rate (2010-2023)} & +6.5\% & +8.2\% & +5.4\% \\ \hline
		\end{tabularx}	
		
	\end{table*}
	
	To systematically analyze the intellectual structure of social network research, this study employs three interconnected bibliometric approaches: \textbf{performance analysis}, \textbf{science mapping}, and \textbf{network analysis}. Each method provides distinct yet complementary insights into the research landscape of the selected journals.
	
	\textbf{Performance analysis} quantitatively evaluates research productivity and impact by examining publication trends, citation metrics, and contributions from influential authors, institutions, and countries. This approach highlights the historical growth of the field and identifies its most prolific contributors.
	
	\textbf{Science mapping} explores the thematic and conceptual structure of the field by uncovering research clusters, keyword co-occurrence patterns, and intellectual influences. Using techniques such as co-citation analysis, bibliographic coupling, and topic modeling, science mapping reveals the knowledge base and thematic evolution of social network research.
	
	\textbf{Network analysis} investigates collaboration patterns among authors, institutions, and countries. By analyzing co-authorship networks, institutional collaborations, and international partnerships, this approach sheds light on the social and structural dynamics of knowledge production in the field.
	
	By systematically integrating these three analytical components, this study provides a comprehensive and structured overview of social network research. The results will:
	\begin{itemize}
		\item Reveal historical and emerging research trends, highlighting shifts in dominant themes and methodological approaches.
		\item Identify key contributors and influential works, offering insights into the most impactful papers, prolific scholars, and leading institutions.
		\item Map collaborative networks, examining how scholars, institutions, and countries interact and contribute to the field.
	\end{itemize}
	
	This multi-faceted approach not only quantifies research impact but also contextualizes the development of ideas, theories, and methodologies in social network research. The following sections provide a detailed description of each bibliometric technique, including its purpose, methodology, data requirements, and analytical tools.
	
	To ensure a comprehensive and accurate dataset for bibliometric analysis, we retrieved publication records from the Web of Science (WoS), limiting our selection to articles published in \textit{Social Networks}, \textit{Journal of Complex Networks}, and \textit{Network Science}. The search was structured to include only records explicitly affiliated with these journals, avoiding misclassification from unrelated sources. The full query links used for data extraction are provided below:
	\begin{itemize}
		\item \url{https://www.webofscience.com/wos/woscc/summary/8b9cb9e7-5920-4a8c-8957-1f45746eb38f-01449729f7/relevance/1}
		\item 	\url{https://www.webofscience.com/wos/woscc/summary/b70d5df8-5cd9-4064-8329-390221e5fcc0-014497387b/relevance/1}
		\item 	\url{https://www.webofscience.com/wos/woscc/summary/cfb2985d-4457-4214-85c1-4ee6224ffecb-014497425b/relevance/1}
	\end{itemize}
	
	Following data retrieval, a preprocessing step was conducted to refine the dataset. We verified that each record belonged to one of the three journals and removed any erroneous entries. Records with missing or incomplete metadata—such as missing publication years, authorship details, or incorrectly formatted citations—were excluded. These measures ensured that our analysis accurately reflected the scholarly output in these journals without distortions from irrelevant or unreliable data.
	
	
	\section{Performance Analysis}\label{Performance Analysis}
	
	Performance analysis in bibliometrics provides an overview of research productivity, citation impact, and contributions of authors, institutions, and countries. It helps quantify the most influential publications, prolific researchers, and overall research trends in \textit{Social Networks}, \textit{Journal of Complex Networks}, and \textit{Network Science}.
	
	\subsection{Publication Trends in Social Network Research}
	
	\begin{figure}[htbp]
		\centering
		\caption{Publication Trends}\label{fig.fig1}
		\includegraphics[width=\columnwidth]{images/Record Count Proportion.pdf}
	\end{figure}
	
	Examining the publication trajectories of \textit{Social Networks}, \textit{Journal of Complex Networks}, and \textit{Network Science} provides insight into the evolution of social network research. Figure \ref{fig.fig1} illustrates how research output has changed over time, highlighting the growth and specialization of the field.
	
	\subsubsection*{Expansion and Institutionalization of Social Network Research}
	
	\textit{Social Networks}, the longest-running journal in the field, has published research continuously since its founding in 1978. For decades, it served as the primary outlet for social network analysis, maintaining stable publication volumes. However, since 2010, a notable increase in output reflects the growing influence of computational methods and empirical applications. This trend suggests a broadening of the field as new methodologies and interdisciplinary collaborations expand its research scope.
	
	The establishment of \textit{Journal of Complex Networks} in 2013 marked the rising prominence of mathematical and computational approaches to network analysis. Initially publishing at a modest rate, its output grew rapidly after 2015, nearing that of \textit{Social Networks} by 2020. This growth highlights the emergence of complex network research as a distinct subfield, attracting contributions from applied mathematics, computer science, and physics.
	
	\textit{Network Science}, introduced in 2014, maintains a comparatively lower publication volume, indicating its focus on a specialized academic community. Emphasizing foundational principles and interdisciplinary perspectives, the journal has experienced steady growth but remains the smallest of the three in annual output, reinforcing its niche within the broader field of network science.
	
	\subsubsection*{Publication Growth and Recent Trends}
	
	Between 2016 and 2022, publication activity in all three journals increased significantly, reflecting the growing influence of network science. \textit{Social Networks} exceeded 100 publications per year, while \textit{Journal of Complex Networks} and \textit{Network Science} followed similar trajectories on a smaller scale. This growth aligns with the rise of big data, computational social science, and machine learning, which have reinforced the prominence of network-based methodologies.
	
	Since 2022, publication volumes have stabilized or declined, potentially signaling research saturation or database indexing delays. Further analysis is needed to determine whether this trend reflects a plateau or a shift in thematic focus.
	
	\subsubsection*{Thematic Differentiation Among Journals}
	
	The trajectories of these journals highlight the thematic specialization within social network research. \textit{Social Networks} remains the primary outlet for applied and empirical studies, emphasizing sociology, organizational research, and human behavior. \textit{Journal of Complex Networks} focuses on computational and mathematical approaches, while \textit{Network Science} bridges theoretical and interdisciplinary perspectives.
	
	This specialization reflects the diversification of network science into distinct subfields, each catering to specific research communities. Further analysis of citation networks and thematic clustering will clarify how these journals interact and shape the field over time.
	
	\subsection{Prolific Authors}\label{Prolific Authors}
	
	This section investigates the contributions of the most prolific authors within the domain of social network research, across three prominent journals: \textit{Social Networks}, \textit{Journal of Complex Networks}, and \textit{Network Science}. By analyzing the publication records, we uncover the significant roles these individuals play in shaping the field and defining its intellectual boundaries.
	
	\begin{figure*}[htbp]
		\centering
		\includegraphics[width=\textwidth]{images/Top 30 Authors by Publication Count Across Three Journals.pdf}
		\caption{Prolific Authors}
		\label{fig.fig2}
	\end{figure*}
	
	\subsubsection*{Author Contributions and Research Orientations}
	
	The distribution of publications among the most prolific authors reveals distinct preferences and specializations that align closely with each journal's thematic and methodological emphases. Notably, authors like \textbf{Doreian P}, \textbf{Borgatti SP}, and \textbf{Everett MG} dominate in \textit{Social Networks}, suggesting their research aligns with traditional social network analysis, which often focuses on sociological applications and network dynamics within communities.
	
	In stark contrast, \textbf{Barabási AL} and \textbf{Porter MA} show a strong inclination towards \textit{Journal of Complex Networks} and \textit{Network Science}, indicating their work's alignment with more computational and theoretical approaches. These journals typically attract studies centered on complex systems, network theory, and often interdisciplinary approaches that draw from physics, computer science, and biology.
	
	\subsubsection*{Patterns of Specialization and Cross-Journal Engagement}
	
	The limited cross-journal publication by authors may stem from the distinct academic cultures and publication strategies inherent in their areas of expertise. It is reasonable to hypothesize that:
	\begin{itemize}
		\item \textbf{Empirical and Applied Research Focus:} Authors publishing predominantly in \textit{Social Networks} might prioritize empirical data and real-world applications, which aligns with the journal's aim to influence practical and policy-related outcomes.
		\item \textbf{Theoretical and Computational Focus:} Conversely, authors like \textbf{Porter MA} and \textbf{Barabási AL} engage with journals like \textit{Journal of Complex Networks} and \textit{Network Science} due to their interest in developing new theoretical frameworks and computational models that may not align with the more applied nature of \textit{Social Networks}.
	\end{itemize}
	
	This specialization underlines a broader academic phenomenon where researchers often become siloed within their disciplinary boundaries, occasionally leading to challenges in interdisciplinary research dissemination. The fact that very few authors publish across all three journals suggests a significant opportunity for promoting interdisciplinary research, which could bridge gaps between empirical and computational studies.
	
	\subsubsection*{Implications for the Field}
	
	This segmentation of publishing within specific journals reflects broader intellectual trends and may suggest potential barriers to interdisciplinary research. While \textit{Social Networks} continues to draw empirical research, the theoretical insights from \textit{Journal of Complex Networks} and \textit{Network Science} could enrich these empirical findings, and vice versa.
	
	The presence of a small but notable group of cross-journal contributors offers a glimpse into the potential for more integrated research approaches. By fostering interdisciplinary contributions, the field could leverage computational models to enhance empirical research, ultimately leading to more robust findings that can advance the understanding of network processes across different domains.
	
	In subsequent sections, we will delve deeper into how these publication patterns reflect the evolving landscape of network research and what they imply about the integration of methodological innovations across disciplines.
	
	
	
	
	\subsection{Institutional Contributions to Social Network Research}\label{Institutional Contributions to Social Network Research}
	
	This section examines the leading institutional contributors to social network research, focusing on their publication records across three key journals: \textit{Social Networks}, \textit{Journal of Complex Networks}, and \textit{Network Science}. By analyzing the top affiliations in each journal, we gain insight into dominant institutions, regional patterns, and the differing research orientations reflected in these publication venues.
	
	\subsubsection*{Institutional Influence and Regional Disparities}
	
	A clear pattern emerges in the institutional distribution of social network research. The \textit{University of California System} is the most prolific contributor, with substantial publication output across all three journals. Its presence is most pronounced in \textit{Social Networks}, where it accounts for the largest institutional share, but it also maintains a notable footprint in \textit{Journal of Complex Networks} and \textit{Network Science}. This broad engagement underscores its commitment to both empirical and computational network science.
	
	Beyond the University of California System, North American institutions dominate contributions to \textit{Social Networks}, reinforcing the journal’s strong ties to sociology and applied network analysis. Universities such as \textit{Pennsylvania Commonwealth System of Higher Education}, \textit{University of Pittsburgh}, and \textit{University of California Irvine} rank among the most frequent contributors. Meanwhile, European universities, particularly \textit{University of Groningen} and \textit{University of Oxford}, play a central role, highlighting the journal’s reach beyond the United States.
	
	\subsubsection*{Theoretical Focus in Journal of Complex Networks}
	
	In contrast, \textit{Journal of Complex Networks} features a stronger presence of European institutions, reflecting its emphasis on mathematical and algorithmic approaches to network science. The leading contributors include \textit{University of Oxford}, \textit{Centre National de la Recherche Scientifique (CNRS)}, and \textit{University of London}, institutions known for their focus on theoretical modeling and complexity science. Many of these affiliations maintain collaborations with physics and computer science departments, further reinforcing the journal’s orientation toward formal network analysis.
	
	\subsubsection*{Interdisciplinary Engagement in Network Science}
	
	\textit{Network Science} presents a more interdisciplinary institutional composition, incorporating both theoretical and applied perspectives. It attracts contributions from leading research universities, including \textit{Harvard University}, \textit{Indiana University}, and \textit{Central European University}, each of which has established itself as a center for computational and quantitative social science. Additionally, institutions such as \textit{The Santa Fe Institute} and \textit{CNRS} are well represented, reflecting the journal’s emphasis on interdisciplinary and fundamental research in network theory.
	
	\subsubsection*{Institutional Overlap and Specialization}
	
	While a few institutions maintain a presence across all three journals, most exhibit specialization in either applied or theoretical network research. Universities such as \textit{Oxford} and \textit{California} contribute broadly, spanning both empirical and computational network studies. However, others, like \textit{CNRS} and \textit{Harvard}, are more concentrated in \textit{Journal of Complex Networks} and \textit{Network Science}, respectively, signaling a stronger focus on formal network methodologies. The relative lack of institutional overlap suggests that, despite their shared focus on network research, these journals cater to distinct scholarly communities.
	
	\subsubsection*{Implications for the Field}
	
	The institutional landscape of social network research reflects both regional and disciplinary distinctions. \textit{Social Networks} remains closely linked to North American universities with strong traditions in empirical network studies, while European institutions lead contributions to \textit{Journal of Complex Networks} and \textit{Network Science}, reinforcing their prominence in complexity science and theoretical research. The presence of highly specialized institutions such as \textit{The Santa Fe Institute} highlights the growing role of interdisciplinary approaches, while increasing contributions from regions outside North America and Europe—such as \textit{Universidade de São Paulo}—signal a gradual globalization of the field.
	
	\begin{table}[htbp]
		\centering
		\caption{Top 10 Affiliations per Journal}
		\label{table.tab1}
		\resizebox{\columnwidth}{!}{
			\begin{tabular}{lcl}
				\toprule
				Affiliations & Record Count & Journal \\
				\midrule
				UNIVERSITY OF CALIFORNIA SYSTEM               & 171 & Social Networks \\
				UNIVERSITY OF CALIFORNIA IRVINE               & 86 & Social Networks \\
				PENNSYLVANIA COMMONWEALTH SYSTEM OF HIGHER ED & 73 & Social Networks \\
				UNIVERSITY OF GRONINGEN                       & 66 & Social Networks \\
				UNIVERSITY OF OXFORD                          & 55 & Social Networks \\
				UNIVERSITY OF PITTSBURGH                      & 52 & Social Networks \\
				UTRECHT UNIVERSITY                            & 47 & Social Networks \\
				UNIVERSITY OF CHICAGO                         & 41 & Social Networks \\
				UNIVERSITY OF MELBOURNE                       & 41 & Social Networks \\
				UNIVERSITY OF SOUTH CAROLINA COLUMBIA         & 41 & Social Networks \\
				UNIVERSITY OF OXFORD                          & 30 & Journal of Complex Networks \\
				UNIVERSITY OF CALIFORNIA SYSTEM               & 21 & Journal of Complex Networks \\
				HARVARD UNIVERSITY                            & 20 & Network Science \\
				UNIVERSITY OF CALIFORNIA SYSTEM               & 18 & Network Science \\
				CENTRE NATIONAL DE LA RECHERCHE SCIENTIFIQUE  & 18 & Journal of Complex Networks \\
				UNIVERSITY OF LONDON                          & 16 & Journal of Complex Networks \\
				INDIANA UNIVERSITY BLOOMINGTON                & 15 & Network Science \\
				INDIANA UNIVERSITY SYSTEM                     & 15 & Network Science \\
				NORTHEASTERN UNIVERSITY                       & 15 & Network Science \\
				CENTRAL EUROPEAN UNIVERSITY                   & 14 & Network Science \\
				UNIVERSIDADE DE SAO PAULO                     & 14 & Journal of Complex Networks \\
				CENTRE NATIONAL DE LA RECHERCHE SCIENTIFIQUE  & 13 & Network Science \\
				THE SANTA FE INSTITUTE                        & 13 & Journal of Complex Networks \\
				HARVARD MEDICAL SCHOOL                        & 12 & Network Science \\
				UNIVERSITY OF OXFORD                          & 12 & Network Science \\
				MAX PLANCK SOCIETY                            & 11 & Journal of Complex Networks \\
				UNIVERSITY OF CALIFORNIA LOS ANGELES          & 11 & Journal of Complex Networks \\
				UNIVERSITY OF TRENTO                          & 11 & Journal of Complex Networks \\
				UNIVERSITY OF GRONINGEN                       & 10 & Network Science \\
				MASSACHUSETTS INSTITUTE OF TECHNOLOGY MIT     & 10 & Journal of Complex Networks \\
				\bottomrule
		\end{tabular}}
	\end{table}

	\subsubsection*{Interpretation of Citation and Publication Metrics}
	
	To assess the research impact and scholarly influence of the three selected journals, we computed key bibliometric indicators, summarized in Table \ref{table.tab3}. These metrics provide a structured evaluation of citation patterns, research productivity, and author contributions. Specifically, we analyze the number of cited publications (NCP), the proportion of cited publications (PCP), and the average citations per cited publication (CCP) to assess citation reach. Additionally, we employ widely used impact measures such as the h-index, g-index, and i-index at multiple citation thresholds to capture both depth and breadth of influence within the field. These indicators collectively offer insights into the relative standing and thematic focus of each journal in social and complex network research.
	\begin{table*}[htbp]
		\scriptsize
		\caption{Citation and Publication Related Metrics}\label{table.tab3}
		\begin{tabularx}{\textwidth}{l|r|r|r|r|r|r|r|r}
			\toprule
			Journal & NCP & PCP & CCP & h-index & g-index & i-10 index & i-100 index & i-200 index \\
			\midrule
			Social Networks & 1462 & 0.939000 & 64.510000 & 126 & 264 & 1004 & 170 & 69 \\
			Network Science & 252 & 0.834400 & 14.680000 & 23 & 52 & 78 & 6 & 1 \\
			Journal of Complex Networks & 431 & 0.786500 & 18.500000 & 34 & 78 & 138 & 10 & 5 \\
			\bottomrule
		\end{tabularx}
	\end{table*}
	\textit{Social Networks} demonstrates the highest citation influence, with 1,462 cited publications (NCP) and a proportion of cited publications (PCP) of 93.9\%. This indicates a strong citation uptake, reaffirming its status as a leading outlet for social network research. The average citations per cited publication (CCP = 64.51) is considerably higher than in the other two journals, suggesting that publications in \textit{Social Networks} tend to be more widely cited. The high h-index (126) and g-index (264) further highlight the substantial impact of its articles.
	
	In contrast, \textit{Network Science} and \textit{Journal of Complex Networks}, both more recent journals, exhibit lower citation metrics. \textit{Network Science} has a PCP of 83.4\% and a CCP of 14.68, indicating that while most of its articles receive citations, they accumulate fewer per publication. Similarly, \textit{Journal of Complex Networks} reports a PCP of 78.7\% and a CCP of 18.5, reflecting its role in a more specialized computational and theoretical niche.
	
	The distribution of highly cited publications aligns with these patterns. \textit{Social Networks} has 1,004 papers exceeding 10 citations (i-10), with 170 surpassing 100 citations (i-100) and 69 exceeding 200 citations (i-200). This far surpasses the counts in \textit{Network Science} (i-100 = 6, i-200 = 1) and \textit{Journal of Complex Networks} (i-100 = 10, i-200 = 5), reinforcing its dominant position in the field.
	
	Overall, these metrics reflect the distinct roles of each journal: \textit{Social Networks} as the primary venue for empirical and applied social network studies, \textit{Journal of Complex Networks} as a hub for mathematical and computational approaches, and \textit{Network Science} as an interdisciplinary bridge. The citation trends highlight the field’s evolution, with computational and theoretical research gaining visibility while traditional empirical studies maintain the highest impact.

	
	\section{Science Mapping}
	
	In this section, we'll perform science mapping on the data we retrieved. 
	
	\bibliographystyle{apalike}
	\bibliography{ref}
	
\end{document}
